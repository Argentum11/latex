\documentclass[11pt]{article}
\pagestyle{empty}
\parindent 0px

\begin{document}
superscripts
$$2x^{3x+4}$$

subscripts
$$x_{1_2}$$
$$a_0, a_1,\cdots a_n$$
$$a_0, a_1,\ldots a_n$$

Greek letters
$$\pi$$
$$\alpha$$

Trig functions
$$y=\sin x$$
$$y=\arccos x$$

Log functions
$$y=\log x$$
$$y=\log_{5} x$$
$$y=\ln x$$

Roots
$$\sqrt{100}$$
$$\sqrt[3]{100}$$
$$\sqrt{   1+\sqrt{x}   }$$

Fractions
$$\frac{3}{7}$$
About $\displaystyle \frac{3}{7}$ is for tax.About $\frac{3}{7}$ is for tax.\\[6pt]
About $\frac{3}{7}$ is for tax.(Vertial space)
$$\frac{\sqrt{x+1}}{\sqrt{x+3}}$$
$$\frac{1}{   1+\frac{1}{x}   }$$

Brackets\\
The set A is defined as $\{1, 2, 3\}$\\
To make brackets automatically fit the size of the elements inside, use \textbackslash left and \textbackslash right
$$4(\frac{1}{x+\frac{2}{3}})$$
$$4\left(\frac{1}{x+\frac{2}{3}}\right)$$
$$4\left\langle\frac{1}{x+\frac{2}{3}}\right\rangle$$
But \textbackslash left and \textbackslash right must be in pairs, to hide one of them, use \textbackslash (left/right).
$$\left.\frac{dy}{dx}\right|_{x=1}$$


\end{document}